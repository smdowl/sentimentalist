% Default mode is landscape, which is what we want, however dvips and
% a0poster do not quite do the right thing, so we end up with text in
% landscape style (wide and short) down a portrait page (narrow and
% long). Printing this onto the a0 printer chops the right hand edge.
% However, 'psnup' can save the day, reorienting the text so that the
% poster prints lengthways down an a0 portrait bounding box.
%
% 'psnup -w85cm -h119cm -f poster_from_dvips.ps poster_in_landscape.ps'

\documentclass[a0]{a0poster}
% You might find the 'draft' option to a0 poster useful if you have
% lots of graphics, because they can take some time to process and
% display. (\documentclass[a0,draft]{a0poster})
\usepackage[latin1]{inputenc}
\usepackage{tikz}
\usetikzlibrary{shapes,arrows}
\pagestyle{empty}
\renewcommand{\d}{\mathrm{d}}
\newcommand{\sgn}[1]{\mathop{\mathrm{sgn}}#1}
\newcommand{\bu}{\mathbf{u}}
\newcommand{\bx}{\mathbf{x}}
\newcommand{\br}{\mathbf{r}}
\newcommand{\ds}{\mathrm{d}s}
\newcommand{\ie}{\textit{i.e.}}
\setcounter{secnumdepth}{0}
\newcommand{\comment}[1]{}

% The textpos package is necessary to position textblocks at arbitary 
% places on the page.
\usepackage[absolute]{textpos}

% Graphics to include graphics. Times is nice on posters, but you
% might want to switch it off and go for CMR fonts.
\usepackage{tikz}
\usepackage{graphics}
\usepackage{wrapfig,helvet}
\usepackage{amsmath}


% These colours are tried and tested for titles and headers. Don't
% over use color!
\usepackage{color}
\definecolor{DarkBlue}{rgb}{0.1,0.1,0.5}
\definecolor{Red}{rgb}{0.9,0.0,0.1}
\definecolor{headingcol}{rgb}{0.5,0.7,1}
%\definecolor{boxcol}{rgb}{0.3,0.8,0.1}

% see documentation for a0poster class for the size options here
\let\Textsize\normalsize
\def\Head#1{\noindent\hbox to \hsize{\hfil{\LARGE\color{DarkBlue}\sf #1}}\bigskip}
\def\LHead#1{\noindent{\LARGE\color{DarkBlue}\sf #1}\bigskip}
\def\Subhead#1{\noindent{\large\color{DarkBlue}\sf #1}\bigskip}
\def\Title#1{\noindent{\VeryHuge\color{Red}\bf\sf #1}}

\TPGrid[40mm,40mm]{23}{12}  % 3 cols of width 7 plus 2 gaps width 1

\parindent=0pt
\parskip=0.5\baselineskip

\makeatletter							%Needed to include code in main file
\renewcommand\@maketitle{%
\null									%Sets position marker
{
\color{headingcol}\sffamily\VERYHuge		%Set title font and colour
\@title \par}%
\vskip 0.6em%
{
\color{white}\sffamily\LARGE				%Set author font and colour
\lineskip .5em%
\begin{tabular}[t]{l}%
\@author
\end{tabular}\par}%
\vskip 1cm
\par
}
\makeatother

\title{Sample UCL-styled A0 scientific poster \LaTeX}

\author{Shaun Dowling, Alessandro Ialongo, Andrey Levushkin, Matthieu Louis\\ University College London}

\begin{document}
%----------------------------------------------------------------------%
%           Title bar: across all 21 columns                           %
%----------------------------------------------------------------------%
\begin{textblock}{23}(0,0)
\vspace*{-48mm}\hspace*{-42mm}%
\includegraphics{ucl_bar_black.eps}
\begin{minipage}{1191mm}		%Minipage for title contents
\vspace{-20cm}
\maketitle
\end{minipage}
\end{textblock}

%%%%%%%%%%%%%%%%%% Will need to shift all other content down a bit %%%%%

%----------------------------------------------------------------------%
%           First column.                                              %
%----------------------------------------------------------------------%
\begin{textblock}{7}(0,2.4)
\Head{Introductory segment}

\sf % Selects sans serif family: part of the UCL corporate image!
Lorem ipsum dolor sit amet, consectetuer adipiscing elit. Vestibulum justo. Praesent leo. Sed consectetuer. Aenean pretium, diam quis mattis porttitor, elit velit scelerisque sapien, sed convallis ipsum lectus non neque. Morbi in mi eu neque luctus scelerisque. Curabitur odio. Mauris a mi. Aenean iaculis erat vel sapien. Curabitur nulla velit, feugiat quis, imperdiet sit amet, vestibulum ac, diam. Suspendisse a metus. Pellentesque vulputate venenatis eros. In auctor, eros nec sodales faucibus, erat nisl facilisis nisl, ut rutrum nisl nunc eu est. Quisque eget ante at nunc varius ultrices.

\bigskip
\hrule
\end{textblock}

\begin{textblock}{7}(0,4.52)

\Head{First Piece of Content}

\sf 
It is worth fiddling around a bit with positioning of the text blocks to get the spacing even. I like a vertical gap of about 0.4 ``block units'' between the horizontal bar at the end of one block and the beginning of the next. There are contruction lines at the end of the \TeX\ file to help with this.



% Define block styles
\tikzstyle{decision} = [diamond, draw, fill=blue!20, 
    text width=4.5em, text badly centered, node distance=3cm, inner sep=0pt]
\tikzstyle{block} = [rectangle, draw, fill=blue!20, 
    text width=5em, text centered, rounded corners, minimum height=4em]
\tikzstyle{line} = [draw, -latex']
\tikzstyle{cloud} = [draw, ellipse,fill=red!20, node distance=3cm,
    minimum height=2em]
    
\begin{tikzpicture}[node distance = 2cm, auto]
    % Place nodes
    \node [block] (init) {Tweets};    
    \node [block, below of=init, left] (feat1) {Feature Extraction};
    \node [block, below of=init, right] (feat2) {Global Feature Extraction};
    %\node [block, left of=evaluate, node distance=3cm] (update) {update model};
    \node [block, below of=feat1, node distance=3cm] (ml) {Machine Learning};
    \node [block, below of=ml] (eva) {Evaluation};
    % Draw edges
    \path [line] (init) -- (feat1);
    \path [line] (init) -- (feat2);
    \path [line] (feat1) -- (ml);
    %\path [line] (feat2) -| node [near start] {yes} (update);
    \path [line] (feat2) -- (ml);
    \path [line] (ml) -- (eva);
    %\path [line,dashed] (expert) -- (init);
\end{tikzpicture}


Figures (and labels using the \LaTeX\ picture environment) work as you would expect. 

\bigskip
\hrule
\end{textblock}

\begin{textblock}{7}(0,8.20)
\Head{Feature Extraction}

\sf
The key objective of this project is to use information contained within Twitter posts to predict the mood of the markets towards certain stocks. 
Tweets contain a great deal of information, including the text itself, linking between users, linking to entities (either explicitly view a tag or plain text) and network effect via re-tweets.

In their raw form, Tweets are not convenient for inference. As such we will focus a great deal of our time distilling the information contained within the Tweets into a form that we can use easily, ensuring all important aspects of the Tweets are preserved.
This approach gives a nice layer of abstraction between the Information Retrieval and the Machine Learning steps in the pipeline. 
We will be gathering these features using two independent MapReduce jobs.


\Subhead{Global Statistics}

\sf
Firstly, we will run a global job to gather statistics that depend on the entire dataset (for example tf/idf).
The statistics available and the mannerr in which they are gathered will be a significant area of experimentation.

Some statistics that we plan to start with are:
\begin{itemize}
\item user average sentiment - to help us determine if someone's good opinion of an entity is only because they are generally positive. This could be global or regarding a specific company.
\item inverse document frequency - to be used with term frequency within tweets to potentially highlight particularly significant phrases
\item cliques of users (strongly connected groups) - experiment with the effect of an entire connected group having a particular sentiment or sudden change of sentiment
\end{itemize}

\Subhead{Extractor}

Having gathered global statistics, we can split with another map the Tweets by time segment such that the reducer can then create a full feature vector for each time slice as we see fit.
Architecturally, there will be an abstract Extractor component within the reducer that is given the full list of tweets to process.
This component will be the central component of change in order to explore different options in feature extraction.
The Extractor will also have access to whatever global statistics produced previously in order to compute more complex features

\sf
Key local features that initially be calculated will be things like
\begin{itemize}
\item entity mentions - a base feature which will be joined with a number of other to attribute features to the company whose stock we are interested in
\item sentiment - tied to a specific entity or connected group of users
\item word frequencies - again tied to a entity
\item user ID - which users are mentioning a company
\item number of re-tweets - potentially linked to other statistics so that they can be weighted accordingly
\end{itemize}

\bigskip
\hrule
\end{textblock}

%----------------------------------------------------------------------%
%           Second column.                                             %
%----------------------------------------------------------------------%

\begin{textblock}{7}(8,5.25)

\Head{Machine Learning}

\sf
The feature extraction process provides the raw material on which to construct models to explain the data and to formulate predictions about stock prices given new Tweets. In our project we will consider three main statistical models to uncover the patterns in the data:
\begin{itemize}
\item Linear Regression
\item Support Vector Machine (SVM)
\item Gaussian Process
\end{itemize}
Each of these models will make use of a portion of the available Twitter data (between 70\% and 90\% of the data, ordered chronologically) in combination with stock price data to extract the optimal parameters (according to a loss function). The parameterised models will then be used to predict the more recent performance of the relevant stocks given the remaining portion of the Twitter data. These predictions will then be evaluated against the real-world performance. 

\Subhead{Linear Regression}
We will use regularised linear regression with the MSE (mean squared error) loss function (ridge regression) to give us a simple baseline on our predictive performance. With this simple model, we hope to find a linear relation (i.e. linear coefficients) between the features previously extracted and the stock prices. The (Tikhonov) regularisation will be parameterised by a lambda value which will determine the extent to which more complex (larger) coefficients will be penalised in our MSE loss function.

\Subhead{SVM}
When predicting simple increase or decreases in stock prices (see evaluation), the problem becomes a classification rather than a regression one. Thus we can use a support vector machine (experimenting with possible kernels and respective parameters) to fit a hyperplane in the feature space between the time splits in the Twitter data that corresponded to increases in the relevant stock prices from those that corresponded to decreases. 

\Subhead{Gaussian Process}
For the regression task we will also attempt to describe the stock market as a Gaussian interaction of multiple samples defined by a suitable (kernel) covariance matrix. Defining this matrix will allows us to model periodicities, and the decay of correlation between adjacent samples. Out of the three, this model is the most elaborate as it allows for the greatest flexibility on the relation between the features and the stock prices. In fact the kernel covariance can encode very complex nonlinear relations between features and data-points.  This requires also a higher degree of cross-validation to select the most effective kernel and its hyperparameters.

\bigskip
\hrule
\end{textblock}

%----------------------------------------------------------------------%
%           Third column.                                              %
%----------------------------------------------------------------------%
\begin{textblock}{7}(16,2.4)

\Head{Performance Evaluation}

\Subhead{ML algorithm perfomance}

\sf
To demonstrate the effectiveness of our learning algorithms we proceed by spiting the data into testing and training sets. Than we predict the future stock price using the training set and compute the mean squared error of the prediction using the test set.

By carrying out the above procedure with different sets of test and training data we will be able to establish the predictive power of our model as well as determine how far into the future we will be able to make effective predictions.

If the system will be used for training than the complex regression problem can be reduced to a simpler classification problem. Instead of predicting the exact price we will instead classify whether stock is going to go up or down after a specific amount of time. When evaluating this approach we intend to use the number of misclassification as a metric for determining the predictive power of our algorithm.


\bigskip

\Subhead{Back-testing}

While mean square error in prediction is useful in evaluating algorithm effectiveness, low mean square error does not directly translate into trading performance.  

The simple price model will need be further extended to incorporate stock liquidity and ensure that gains can be realised. Further optimisation is possible by incorporating trading fees as well as liquidity rebates to ensure that the system not only maximises prediction power but also profitability.


\bigskip
\hrule
\end{textblock}


\begin{textblock}{7}(16,6.85)

\Head{Discussion and Conclusions}

\sf


Proin dignissim nunc in nulla. Vivamus non leo. Nulla ultrices tempor dui. Curabitur nec metus. Aliquam sed libero. Cras orci odio, molestie a, suscipit in, placerat vel, nunc. Vestibulum congue, nunc in faucibus scelerisque, ante tortor dapibus nibh, eu tristique diam urna ac magna. Proin cursus. Morbi quam ligula, fermentum vel, dapibus sit amet, euismod nec, justo. Suspendisse potenti. Nulla eu elit. Pellentesque quam est, pretium ac, suscipit sed, viverra id, sapien.
Donec tempor semper tortor. Nunc vulputate. Aliquam vitae metus ut sem euismod accumsan. Duis tincidunt lacus sed ipsum. Class aptent taciti sociosqu ad litora torquent per conubia nostra, per inceptos himenaeos. Quisque nec nisl at erat ornare tempus. Etiam eros odio, ultricies non, hendrerit et, vestibulum in, felis. Vivamus gravida. 

\begin{itemize}
\item First concluding point; we expected this to be so because of the contruction of the argument and blah. 
\item Second concluding point: this one is counterintuitive but we can justify it by reference to the extended discussion above.
\item Third and \textcolor{Red}{most important concluding point}: this is the one we're excited about.
\end{itemize}

\vspace*{4mm} % Sometimes you will have to fudge the final spacing.
\bigskip
\hrule
\end{textblock}

%----------------------------------------------------------------------%
%            Construction lines                                        %

%\begin{textblock}{23}(0,2)\rule{\textwidth}{0.1mm}\end{textblock}
% Shows where the bottom of the header bar should fall.

%\begin{textblock}{23}(0,2.4)\rule{\textwidth}{0.1mm}\end{textblock}
% Shows where the top of each column should start.

%\begin{textblock}{23}(0,12)\rule{\textwidth}{0.1mm}\end{textblock}
% Shows where the bottom of the lowest block in each column should end

%\begin{textblock}{1.5}(6,4.12)\rule{\textwidth}{0.1mm}\end{textblock}
%\begin{textblock}{1.5}(6,4.52)\rule{\textwidth}{0.1mm}\end{textblock}
% Used to find the base of the first block and thus the top of the second.

%\begin{textblock}{1.5}(14,4.85)\rule{\textwidth}{0.1mm}\end{textblock}
%\begin{textblock}{1.5}(14,5.25)\rule{\textwidth}{0.1mm}\end{textblock}
% Same purpose but in the second column.

%\begin{textblock}{1.5}(15,6.05)\rule{\textwidth}{0.1mm}\end{textblock}
%\begin{textblock}{1.5}(15,6.45)\rule{\textwidth}{0.1mm}\end{textblock}
% Same purpose but in the third column.

\end{document}
